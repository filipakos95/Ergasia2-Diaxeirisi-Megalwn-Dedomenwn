\documentclass{article}

%%%%%%%%%%%%%%%%%%%%%%%%packages
\usepackage[utf8]{inputenc}
\usepackage[greek,english]{babel}
\usepackage{alphabeta}
\usepackage{lmodern}
\usepackage[letterpaper,top=2cm,bottom=2cm,left=3cm,right=3cm,marginparwidth=1.75cm]{geometry}
%%%%%%%%%%%%%%%%%%%%%%%%%%%%%%%%%

\title{Second Assignment of Big Data}
\author{Dimitra Adami 1067738\\
	Filippos Mitsos 1019913\\
	Eleni Bonatsou 1067623}
\date{December 2022\\
	University of Patras\\
	MSc Applied Economics and Data Analysis}

\begin{document}
	
	\maketitle
	\newpage
	\tableofcontents
	\appendix
	
	\newpage
	\section{Άσκηση 3.}
	
	
	Αρχικοποιώ μια μεταβλητή, my Data ,με 7 στήλες με αριθμητικές τιμές. Η mydataείναι κενή, Υπάρχουν μόνο στήλες και δεν έχει οριστεί ο αριθμός των γραμμών. Με την  εντολή forπαίρνω 7 τυχαίες τιμές για καθεμία από τις γραμμές που έχουν δημιουργηθεί. Οι τυχαίες αυτές τιμές θα είναι μεταξύ 1 και 10 καθώς στον έλεγχο forέχω ορίσει min = 1 και maχ=10. Κάθε φορά που θα τρέχω το for θα προκύπτουνδιαφορετικές τιμές στο σύνολοδεδομένων. Στην συνεχεία η μεταβλητήrmodelθα κάνει την παλινδρόμησημεταξύ της εξαρτημένης και των ανεξαρτήτωνμεταβλητών (Υ-Χ1….Χ6). Τέλος θα εκτυπώνει την εκτίμηση των συντελεστών της μεταβλητής r model.\\
	
	Πιο συγκεκριμένα στο παράδειγμα μας, στον έλεγχο for έχω το διάστημα 1:4, οπότε θα λάβω υπόψη μου τον σταθερό όρο και τις 3 μεταβλητές Χ1 Χ2 Χ3 , καθώς ο αριθμός των παρατηρήσεων δεν μπορεί να είναι μεγαλύτερος από τον αριθμό των μεταβλητών. Άρα μονό οι 4 πρώτοι παράγοντες χρειάζονται και παίζουν ρόλο στην παλινδρόμηση μας , καθώς θέλω ο πινάκας μου να είναι τετραγωνικός (4 στήλες 4 γραμμές).\\
	
	Εναλλακτικά, θα μπορούσα να πάρω στον έλεγχο ένα διάστημα  1:3 αλλάτώρα θα έπαιρνα τον σταθερόόρο και τις δυο πρώτεςμεταβλητέςΧ1 Χ2. Αντίστοιχα, θα μπορούσα να πάρω και το διάστημα 1:5 με την διαφορά, να πάρωσταθερό όρο και τις πρώτες 4 μεταβλητές χ1 χ2 χ3 χ4 ..\\
	
	\newpage
	
	\section{Άσκηση 4.} \\
	Το άρθρο αναφέρεται σε μια φυλετική σύνθεση μεταξύ των ενόρκων που έχει σημασία στα αποτελέσματα των δικών και συγκεκριμένα στις δίκες κακουργημάτων στην Φλόριντα σε ένα σύνολο δεδομένων μεταξύ 2010 και 2010. Στην έρευνα αυτή εξαιρούνται ορισμένα δικαστήρια μεταξύ των 6-8 ατόμων και επικεντρώνονται σε δίκες με ενόρκους ανάμεσα στις κομητείες της Sarasota και Lake. Συνεπώς, οι δίκες είναι στον αριθμό 712 στις οποίες έχουμε τις εξαρτημένες μεταβλητές χωρίζοντας τους κατηγορούμενους σε καυκάσιους (n=333) και λευκούς (n=379).\\
	
	Οι εξαρτημένες μεταβλητές είναι: i. καταδίκαση κατηγορούμενου για ένα τουλάχιστον αδίκημα, ii. καταδίκαση σε ένα ποσοστό των πέντε πρώτων αδικημάτων στα οποία είχε καταδικαστεί ο κατηγορούμενος, iii. ο ρόλος που φέρει η φυλετική σύνθεση των ενόρκων στις καταδίκες μεταξύ λευκών και καυκάσιων κατηγορούμενων.\\
	
	Οι ανεξάρτητες μεταβλητές είναι: i. κατηγορία για ναρκωτικά, ληστεία, ιδιοκτησία, οπλοκατοχή, σεξουαλική παρενόχληση, δολοφονία, ii. καυκάσιος, iii. Άνδρας-γυναίκα, iv. Αγγλόφωνος-Ισπανόφωνος, v. Οι συνολικές κατηγορίες που έχει ο καθένας.\\
	
	
	Το μοντέλο παλινδρόμησης που χρησιμοποιείται στο άρθρο είναι:\\
	\begin{center}
		
		\large $y_{i} = \beta_0 + \beta_1  X_{1i} + \beta_2 X_{2i} + ... + \epsilon_{i} $
		
	\end{center}
	
	Η μέθοδος παλινδρόμησης που χρησιμοποιείται είναι η OLS (Μέθοδος Ελαχίστων Τετραγώνων) που αποτελεί μια μέθοδο εκτίμησης των συντελεστών Β σε ένα γραμμικό μοντέλο.\\
	
	Ο στόχος της μελέτης του άρθρου είναι η εξήγηση της εξαρτημένης μεταβλήτης. Με τα δεδομένα που προβάλλονται στο άρθρο και αναλύοντάς τα καταλήγει στο συμπέρασμα ότι οι ένορκοι ανεξαρτήτως φυλής ερμηνεύουν τα αποδειστικά στοιχεία σχετικά με τη φυλή του κάθε κατηγορούμενου. Συνεπώς, από τα δεδομένα αυτά καταλήγουμε ότι η αλληλεπίδραση της φυλής του κατηγορουμένου είναι σημαντικός παράγοντας στην άποψη των ενόρκων με αποτέλεσμα να μεταβάλλονται τα αποδεικτικά στοιχεία που υπάρχουν. Αυτός είναι ένας πολύ σημαντικός παράγοντας που χρήζει ιδιαίτερης και μεγάλης προσοχής με αποτέλεσμα να διασφαλιστεί και να εξασφαλιστεί σωστά η δικαιοσύνη σύμφωνα με τον ποινικό κώδικα δικαιοσύνης.\\
	
	
	\newpage
	
	\section{Άσκηση 5} \\
	
	i.  Με παρόμοιο τρόπο με την OLS λειτουργεί και η GD καθώς προσπαθεί να εκτιμήσει τους συντελεστές της παλινδρόμησης που ελαχιστοποιεί μια συνάρτηση κόστους. Πρόκειται για μια επαναληπτική μέθοδο. Σπάνια συναντάμε την ακριβή τιμή του θ . Σε κάθε επανάληψη μειώνει τα σφάλματα.\\
	
	
	Κατανάλωση τροφίμων = 0,17+0,16*εισοδ+0,17*μεγεθος
	
	\vspace {0.5\baselineskip} 
	
	Κάθετος άξονας = συνάρτηση 
	
	\vspace {0.5\baselineskip} 
	
	Οριζόντιος = αριθμός επαναλήψεων 
	
	\vspace {0.5\baselineskip} 
	
	Η τιμή α μας καθορίζει πιο γρήγορα τα θ. Όσο μικρότερο το α τόσες μικρότερες επαναλήψεις.\\
	
	
	
	ii)  Και οι δυο μέθοδοι χρησιμοποιούνται για την εκτίμηση των συντελεστών ενός μοντέλου παλινδρόμησης. Η OLS όσες φορές και να επαναληφθεί θα βγάζει το ίδιο αποτέλεσμα στο β, με το ίδιο σύνολο δεδομένων. Οι εκτιμήσεις της είναι βέλτιστες. Αντίθετα η GD όσες φορές και να επαναληφθεί θα απέχουν κάθε φορά οι εκτιμήσεις. Προσεγγίζει τέλεια το θ αλλά ποτέ ακριβώς αυτά. Είναι επαναληπτική μέθοδος και παρόλο με το ίδιο σύνολο δεδομένων τα θ είναι διαφορετικά καθώς ο αριθμός των επαναλήψεων επηρεάζει τις εκτιμήσεις. \\
	
	
	\section{Άσκηση 7.} \\
	Στην άσκηση αυτή χρησιμοποιούμε την μέθοδο των ελαχίστων τετραγώνων (OLS) των 10 Fold cross validation και εκτιμήσαμε τους συντελεστές του μοντέλου παλινδρόμησης area= b + emp + b*wind + b3*rain + b0. Εκτιμήσαμε το RMSE και για όλο το σύνολο δεδομένων είναι πολύ μεγάλο ενώ όταν πάρουμε το σύνολο δεδομένων για περιοχές μικρότερες των 3,2 εκταρίων το RMSE είναι πολύ μικρότερο. Οπότε το περιορισμένο μοντέλο κάνει την καλύτερη πρόβλεψη.
	
\end{document}